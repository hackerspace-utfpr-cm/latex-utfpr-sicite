% Modelo Desenvolvido no decorrer do Projeto  
% LaTeX para Instituições de Ensino Superior CM 
% Coordenadores: Professor Dr. Adilandri Mércio Lobeiro 
%                Professor Dr. Marco Aurélio Graciotto 
% Bolsista Do Projeto: Discente: Rafael Rampim Soratto 
 
\documentclass{modelo}
\usepackage[brazil]{babel}     
%\usepackage[english]{babel}  
\usepackage[utf8]{inputenc}   
%\usepackage[utf8]{fontenc}
\usepackage{amsmath,amsfonts,amssymb}
\usepackage{graphicx,graphics} 
\usepackage{xcolor} 
%\usepackage{ifthen}
\usepackage{float} 
\usepackage{fancyhdr} 
\usepackage[font={small}]{caption}  
\usepackage{array} 
\usepackage{lastpage}
\usepackage[square]{natbib} 
\bibliographystyle{abntex2}
\setlength\bibhang{0cm}   
\usepackage{verbatim}  
\usepackage{nth} 
\usepackage{layouts} 

\fancypagestyle{plain}{  
	\fancyhf{} 
	%	\fancyheadoffset[LE, RO]{0cm} %ajustando a margem direita 
	\fancyhead[LE,LO]{%     	
		\includegraphics[scale=0.30]{figs/seii}  
		\textcolor{black}{\textbf{\Large \Rnth{8} SEI}} 	 
	}   
	%	\rhead{$\eventedition$ \eventname}
	\fancyhead[RE,RO]{$8^\circ$ Seminário de Extensão e Inovação}
	\fancyfoot[LE,LO]{%
		\footnotesize
		\textbf{página \thepage{} de \pageref{LastPage}}
	}  
	%	\rfoot{\footnotesize{\eventname, \eventedition., \year, \address.}} 
	\fancyfoot[RE,RO]{\footnotesize{Seminário de Extensão e Inovação, 8., 2018, Apucarana, PR.}}
}

\renewcommand{\headrulewidth}{2pt} 
\renewcommand{\footrulewidth}{0.4pt}   
%\setlength{\footskip}{50pt}  
\setlength{\topmargin}{1pt} 
\setlength{\headheight}{50.37pt} 
% \setlength{\footheight}{56.5pt} 
% \setlength{\textheight}{500pt}   
 

%************ INSIRA OS DADOS DO SEU TRABALHO A PARTIR DAQUI *********     
% 
% O resumo deve ter 250 palavras no máximo.  
% e de preferência deve possuir os 4 tópicos: Objetivos, métodos, resultados e conclusão.
 
\newcommand{\resumotexto}  
{  
%\textbf{Objetivos:} \objetivos \
%\textbf{Métodos:} \metodos  
%\textbf{Resultados:} \resultados \
%\textbf{Conclusão:} \conclusao	 
   

Nesse espaço deve ser inserida uma breve discussão inicial sobre o tema da pesquisa, de 100 a 250 palavras e apresentado no formato estruturado, contendo os itens: Objetivos, Métodos, Resultados e Conclusões. Inclua informações como Objetivo: O objetivo deve ser curto, definindo o problema estudado, destacando as lacunas do conhecimento que serão abordadas no artigo. Métodos: As fontes de dados, a população estudada, amostragem, critérios de seleção, procedimentos analíticos, dentre outros, devem ser descritos de forma compreensiva e completa. Resultados: A seção Resultados deve se limitar a descrever os resultados encontrados sem incluir interpretações/comparações. Conclusões: A conclusão dos autores sobre os resultados obtidos e sobre suas principais implicações 
}

% Palavra chave deve ser separada por ponto e ter apenas de 3 a 5 palavras:
 
\newcommand{\palavraschave}{Palavra um. Palavra dois. Palavra três (deve conter de três a cinco palavras-chave, no mesmo idioma do trabalho, separadas entre si por ponto e finalizadas também por ponto } 
 
\newcommand{\agradecimentosTexto}{Esta seção é obrigatória aos trabalhos que receberam bolsa e auxílio financeiro. Deve apresentar os agradecimentos aos principais órgãos de fomento (bolsa e auxílio financeiro), instituições e pessoas que contribuíram para a realização do trabalho. Não exceder 50 palavras e alocá-los antes das referências.}
\newcommand{\kwords}{Keyword one. Keyword two. Keyword tree (as keywords devem ser uma tradução fiel das palavras-chaves, mantendo a formatação destas).} 
\newcommand{\resumoingles}{
 Objectives: The objective should be short, defining the problem studied, highlighting the knowledge gaps that will be addressed in the article. Methods: Data sources, study population, sampling, selection criteria, analytical procedures, among others, must be described in a comprehensive and complete. Results: The Results section should be limited to describing the results without including interpretations/comparisons. Conclusion: The authors' conclusion on the results and their main implications. 
  }   
  
  \newcommand{\tituloingles}{Sample paper to be used for formatting the articles to Seminário de Extensão e Inovação da UTFPR(SEI)} 
  
 
%********************************************************
% Título 
	
\begin{document}

 	
\title{Exemplo de trabalho a ser utilizado como modelo para formatação dos artigos a serem submetidos ao Seminário de Extensão e Inovação da UTFPR(SEI)} 

%{Sample paper to be used as model for format the articles to be submitted to the Seminário de Extensao e Inovação da UTFPR(SEI)}

 
  	 
\author{  
 
% Editar nome e email de autores para o título do texto. 
% Todo autor deve ter universidade e local.   
% Não usar abreviaturas em nomes.
% Para adicionar autores de alguma universidade: 
% \criarautor{Nome}{e-mail} 

\criarautor{Rafael Soratto}{email@hotmail.com}{ Universidade Tecnológica Federal do Paraná - (UTFPR), Campo mourão, PR, Brasil.\\} 
\criarautor{Adilandri Mércio Lobeiro}{adilandri@gmail.com}{ Universidade Tecnológica Federal do Paraná - (UTFPR), Campo mourão, PR, Brasil.\\} 

% Comando para adicionar a universidade dos autores a cima :  
% \criaruniversidade 
%  {Nome da universidade, (SIGLA), Cidade, Estado, Pais }
%\criaruniversidade{ Universidade Tecnológica Federal do Paraná - (UTFPR), Campo mourão, PR, Brasil.}  
 
 
% Outro exemplo:
%\criarautor{Nomex}{emailx},  
%\criarautor{Nomey}{emaily}  

	  
% Indica que os três autores acima são da mesma universidade(Universidade-sigla-estado-país):	 

%\criaruniversidade{Universidade Estadual de Maringá - (UEM), Maringá, PR,   Brasil.}
			 
% Final do Título 
} 
  
%********************************************************  
% Pré-Textual	 

% Nada do Pré-Textual e Pós Textual devem ser alterados!   

\criartitulo  
\thispagestyle{plain}  
\pagestyle{plain}
 
 
	


 
 %*********************************************************
% Textual : 
 
 % Alterar a partir daqui: 
  
 \section{Introdução} 
 
 A estrutura dos artigos é a convencional: Introdução, Métodos, Resultados e Discussão, Considerações Finais (conforme segue ao longo deste documento).
 Notas: As notas devem ser evitadas.
 Os originais devem ser redigidos na ortografia oficial e digitados em folhas de papel tamanho A4. Os trabalhos deverão conter entre seis e oito páginas. O artigo deve ser escrito no programa Word for Windows, em versão 6.0 ou superior. Se você está lendo este documento, significa que você possui a versão correta do programa. Os artigos devem ser enviados em formato .doc ou .docx. Não serão aceitos para avaliação artigos em formato .pdf ou .odt. Sugere-se a utilização deste arquivo para digitar o trabalho.
 Utilizar fonte Calibri 11, espaçamento simples.
 Título e subtítulo (se houver): Manter apenas a inicial da primeira palavra e de nomes próprios em letra maiúscula. Artigos em português devem ter título e subtítulo (se houver) em português e inglês; artigos em inglês devem ter título e subtítulo (se houver) em inglês e português; artigos em espanhol devem título e subtítulo (se houver) em espanhol e português.
 Dados dos autores: a primeira letra de cada nome em maiúscula e o restante em minúsculo. Abaixo do nome do autor deve constar o e-mail e o vínculo institucional, contendo nome da instituição, sigla, cidade, estado e país, separados por vírgula. Não devem ser utilizadas abreviaturas nos nomes dos autores.
 Títulos das sessões: os títulos das sessões devem ser posicionados à esquerda, sem ponto final, adotando o formato apresentado neste modelo.
 
\section{Métodos}
Corpo do texto: o texto deve iniciar na linha abaixo do título das seções.
Aspas devem ser utilizadas somente em citações diretas. Negrito deve ser utilizado para dar ênfase a termos, frases ou símbolos. Itálico deverá ser utilizado apenas para palavras em língua estrangeira (e.g.for exemple).
No caso de uso de alíneas obedecer às seguintes indicações:
\begin{itemize}
\item[a)] cada item de alínea deve ser ordenado alfabeticamente por letras minúsculas seguidas de parênteses;
\item[b)] os itens de alínea são separados entre si por ponto-e-vírgula;
\item[c)] o último item de alínea termina com ponto;
\item[d)]o estilo de alínea constante deste documento pode ser usado para a aplicação automática da formatação correta de alíneas. 
\end{itemize}
 
\subsection{Formatação de Ilustrações e Tabelas} 

Qualquer que seja o tipo de ilustração (desenho, esquema, fluxograma, fotografia, gráfico, mapa, organograma, planta, quadro, retrato, figura, imagem, entre outros) ou tabela, sua identificação aparece na parte superior, precedida da palavra designativa, seguida de seu número de ordem de ocorrência no texto, em algarismos arábicos, travessão e do respectivo título. Após a ilustração ou tabela, na parte inferior, indicar a fonte consultada (elemento obrigatório, mesmo que seja produção do próprio autor), legenda, notas e outras informações necessárias à sua compreensão (se houver).
A ilustração deve ser citada no texto e inserida o mais próximo possível do trecho a que se refere. Ver, por exemplo, a Figura \ref{fig:fig1}. 
\begin{figure}[H]
    \centering 
        \caption{Figura 1}
    \includegraphics[scale=0.5]{figs/utf}
    \label{fig:fig1}
\end{figure}  
  
Tabelas e quadros devem estar centralizados e conter apenas dados imprescindíveis, evitando que sejam muito extensos. Outro item é importante, é que não devem repetir dados já inseridos no texto, ou vice-versa.  
 
\begin{table}[H]
    \centering  
    \caption{Tabela 1}
    \begin{tabular}{|c||c|}  
    \hline
    \cellcolor{teal}Idade     & \cellcolor{teal}Percentual  \\
    Até 20 anos     & 0 \% \\  
    Entre 21 e 30 anos & 10\% \\ 
    Entre 31 e 40 anos & 20\% \\ 
    Entre 41 e 50 anos & 10\% \\ 
    Acima de 50 anos & 40\% \\ 
    \hline
    \end{tabular}
    \label{tab:tab1}
\end{table} 

Caso os dados sejam inéditos e provenientes de uma pesquisa de campo realizada pelos próprios autores do artigo, essa especificação deve constar na fonte, juntamente com o ano da pesquisa de campo.
  
 \subsection{Equações Matemáticas} 
 As equações matemáticas devem aparecer a partir de um deslocamento de 0,5 cm a partir da margem esquerda, em fonte Times New Roman tamanho 10. Números arábicos devem ser usados em equações, inseridos entre parênteses, como ilustrado na Equação \ref{equation:eq1}.  
  
  \begin{equation}
      u = \beta \sin{\left ( \pi x \right )}\frac{\left ( e^{2x}-1 \right ) \left ( e^{y}-1 \right )}{\left ( e^{2}-1 \right ) \left ( e-1 \right )} 
      \label{equation:eq1}
  \end{equation} 
   
 
 \section{Resutados e discussões}
 \section{Citações e Referências} 
 
As citações devem obedecer ao sistema autor-data e estar de acordo com a norma NBR 10520 da Associação Brasileira de Normas Técnicas (ABNT).
Citações diretas de até três linhas acompanham o corpo do texto e se destacam com aspas duplas. Caso o texto original já contenha aspas, estas devem ser substituídas por aspas simples. Exemplos:
Fulano (2008, p. 10) afirma que “[...] é importante a utilização das citações corretamente”.
"Citar trechos de ‘outros autores’ sem referenciá-los, pode ser caracterizado plágio” (FULANO; BELTRANO, 2009, p. 20, grifo do autor).
Para as citações com mais de três linhas, estas devem ser transcritas em parágrafo distinto.  

Exemplo:

\begin{citacao}
Toda citação direta com mais de 03 linhas é considerada uma citação direta longa. A citação com mais de 03 linhas deve ser escrita sem aspas, em parágrafo distinto, com fonte menor e com recuo de 8,0 cm da margem esquerda, terminando na margem direita, conforme ilustrado neste exemplo (FULANO, 2009, p. 150). 
\end{citacao}
A exatidão das referências é de responsabilidade dos autores e devem ser elaboradas de acordo com a NBR 6023 da ABNT.
Todas as referências citadas no texto, e apenas estas, devem ser incluídas ao final, na seção Referências.
As referências devem incluir apenas aquelas centrais e pertinentes à problemática abordada. É, também, desejável que se evite a utilização de livros, priorizando os periódicos como referência.
Todas as obras consultadas que estiverem disponíveis na internet devem ser referenciadas com o endereço eletrônico e data de acesso. 


A exatidão das referências é de responsabilidade dos autores e devem ser elaboradas de acordo com a NBR 6023 da ABNT.
Todas as referências citadas no texto, e apenas estas, devem ser incluídas ao final, na seção Referências.
As referências devem incluir apenas aquelas centrais e pertinentes à problemática abordada. É, também, desejável que se evite a utilização de livros, priorizando os periódicos como referência.
Todas as obras consultadas que estiverem disponíveis na internet devem ser referenciadas com o endereço eletrônico e data de acesso.




%\noindent\fbox{\begin{minipage}{\dimexpr\textwidth-2\fboxsep-2\fboxrule\relax}
%		\centering
%	\begin{tabular}{|c|c|c|}
%		\hline 
%		Seção & Tipografia & Exemplo \\ 
%		\hline 
%		Seções Primárias & Letras Maiúsculas em negrito   & \textbf{1 Seção Primária} \\ 
%		\hline 
%		Seções Secundárias &  &  \\ 
%		\hline 
%		Seções Terciárias &  &  \\ 
%		\hline 
%	\end{tabular} 
%	
%\end{minipage}} 
%
%\begin{mdframed}
%	If A and B are two events that are not mutually exclusive then: 
%	\[ 
%	P(A \cup B) = P(A) + P(B) - P(A \cap B) 
%	\] 
%\end{mdframed}
%
%\bigskip
%\begin{mdframed}[
%	linecolor=teal,linewidth=2pt,% 
%	frametitlerule=true,% 
%	apptotikzsetting={\tikzset{mdfframetitlebackground/.append style={%
%				shade,left color=white, right color=blue!20}}}, 
%	frametitlerulecolor=teal,
%	frametitlerulewidth=1pt, innertopmargin=\topskip,
%	frametitle={Non Mutually Exclusive Events},
%	outerlinewidth=1.25pt
%	]
%	% ----------
%	If A and B are two events that are not mutually exclusive then: 
%	\[ 
%	P(A \cup B) = P(A) + P(B) - P(A \cap B) 
%	\] 
%\end{mdframed}   
% 
% \begin{tcolorbox}[skin=widget,
% 	boxrule=1mm,
% 	coltitle=black,
% 	colframe=teal,
% 	colback=white,
% 	width=(.9\linewidth),before=\hfill,after=\hfill,
% 	adjusted title={ }]
% 	If A and B are two events that are not mutually exclusive then:  
%
% 	$P(A \cup B) = P(A) + P(B) - P(A \cap B)$
% \end{tcolorbox} 
     
 
%*********************************************************** 
% Pós-Textual  

% Pós e Pre Textuais não devem ser alterados, porém as macros que terminam com "Texto" devem ser preenchidas no preâmbulo.

%\titleformat{\section}{\normalfont} 
%{\thesection}{14pt}{\bfseries\Large}   

%\titleformat{\subsection}{\color{teal}\normalfont\bfseries}{\color{teal}\thesubsection}{1em}{}

\newpage  


%\section*{Referências Bibliográficas}  
\bibliography{sei-template-bib} 
 
 
\section*{Agradecimentos}
\thanks{\agradecimentosTexto}    

%\appendix
%\section*{Apêndice, ou complementos (Opcional!)}\label{apendiceA}
%%IMPORTANTE
%%Aqui são incluídas informações complementares, se necessárias; caso contrário RETIRE todo 
%%esse texto. Não esquecer de deixar o comando \end{document} que encontra-se na última linha do arquivo.


\end{document}
